\documentclass{vldb}

\usepackage[utf8x]{inputenc}
\usepackage{url}
\usepackage{listings}
\usepackage{amssymb,amsmath}

\begin{document}

\title{To trust, or not to trust: Highlighting the need for data provenance in mobile apps for smart cities\titlenote{This research is founded by project CIP-ICT-PSP-2012-6 ``IES Cities: Internet Enabled Services for the Cities accross Europe'', funded under ``The Information and Communication Technologies Policy Support Programme''. More info at \url{http://ec.europa.eu/information_society/apps/projects/factsheet/index.cfm?project_ref=325097}}}

\numberofauthors{3} 

\author{
\alignauthor
Mikel Emaldi\\
       \affaddr{Deusto Institute of Technology - DeustoTech}\\
       %\affaddr{Avda. de las Universidades 24}\\
       %\affaddr{Bilbao, Spain}\\
       \email{m.emaldi@deusto.es}
\alignauthor
Oscar Peña\\
       \affaddr{Deusto Institute of Technology - DeustoTech}\\
       %\affaddr{Avda. de las Universidades 24}\\
       %\affaddr{Bilbao, Spain}\\
       \email{oscar.pena@deusto.es}
\alignauthor
Jon Lázaro\\
       \affaddr{Deusto Institute of Technology - DeustoTech}\\
       %\affaddr{Avda. de las Universidades 24}\\
       %\affaddr{Bilbao, Spain}\\
       \email{jlazaro@deusto.es}
\and
\alignauthor
Diego López-de-Ipiña\\
       \affaddr{Deusto Institute of Technology - DeustoTech}\\
       %\affaddr{Avda. de las Universidades 24}\\
       %\affaddr{Bilbao, Spain}\\
       \email{dipina@deusto.es}
\alignauthor
Sacha Vanhecke\\
       \affaddr{Ghent University - iMinds - Multimedia Lab}\\
       %\affaddr{Avda. de las Universidades 24}\\
       %\affaddr{Bilbao, Spain}\\
       \email{sacha.vanhecke@ugent.be}
\alignauthor
Erik Mannens\\
       \affaddr{Ghent University - iMinds - Multimedia Lab}\\
       %\affaddr{Avda. de las Universidades 24}\\
       %\affaddr{Bilbao, Spain}\\
       \email{erik.mannens@ugent.be}
}
 
\date{24 July 2013}

\maketitle

\begin{abstract}
The popularity of smartphones makes them the most suitable devices to ensure access to the services provided by smart cities; furthermore, as one of the main features of the smart cities is the participation of the citizens in their governance, it is not unusual that these citizens generate and share their own data through their smartphones. But, how we can know if are they reliable? How can we know if can a given user and, consequently, the data generated by him/her can be trusted? On this paper, we present how IES Cities platform integrates PROV Data Model and the related PROV-O ontology, allowing the exchange of provenance information about user-generated data in the context of smart cities.

\end{abstract}

\section{Introduction}
\label{sec:introduction}

According to the ``Apps for Smart Cities Manifesto''\footnote{\url{http://www.appsforsmartcities.com/?q=manifesto}}, smart city applications could be sensible, connectable, accessible, ubiquitous, sociable, sharable and visible/augmented. It is not a coincidence that all of these features can be found in a standard smartphone: the popularity of these devices makes them the most suitable device to ensure access to the services provided by smart cities. As one of the main features of the smart cities is the participation of the citizens in their governance, it is not unusual that these citizens generate and share their own data through their smartphones. Reviewing the literature, we can find many examples of apps that deal with user generated data like Urbanopoly \cite{celino_urbanopolysocial_2012}, Urbanmatch \cite{celino_urbanmatch-linking_2012} or the popular apps related to the 311 service in cities like Calgary, Minneapolis, Baltimore or San Diego, all of them available in Google Play\footnote{\url{https://play.google.com}}. The IES Cities project goes one step beyond, providing an entire architecture to allow citizens to develop apps based on Linked Open Data \cite{bizer_linked_2009} provided by government, through user-friendly JSON APIs. All of these works that manage user-generated data have the same worry about these data: are they reliable? How can we know if can a given user and, consequently, the data generated by him/her can be trusted? Recently, the W3C has created the PROV Data Model \cite{moreau_prov-dm:_2012}, for provenance interchange on the Web. This PROV Data Model describes the entities, activities and people involved in the creation of a piece of data, allowing the consumer to evaluate the reliability of the data based on the their provenance information. Furthermore, PROV was deliberately kept extensible, allowing various extended concepts and custom attributes to be used. For example, the Uncertainty Provenance (UP) \cite{de_nies_modeling_2013} set of attributes can be used to model the uncertainty of data, aggregated from heterogeneously divided trusted and untrusted sources, or with varying confidence. On this paper, we present how IES Cities platform integrates PROV Data Model and the related PROV-O ontology \cite{lebo_prov-o:_2013} and Uncertainty Provenance set of attributes, allowing the exchange of provenance information about user-generated data in the context of smart cities. The final aim is to enrich the knowledge gathered about a city not only with government-provided or networked sensors’ provided data, but also with high quality and trustable data coming from the citizens themselves.

The remaining of the paper is organized as follows: in Section \ref{sec:state_of_the_art} the current state of the art on apps that deal with user data in the context of smart cities and on provenance data representation is presented. Section \ref{sec:ies_cities} outlines the main concepts about IES Cities project. Sections \ref{sec:provenance} and \ref{sec:use_cases} describes the metrics to calculate the reliability of the data and its semantic representation, through a use cases, respectively. Finally, in Section \ref{sec:conclusions} the conclusions and the future work are presented.

\section{Related Work}
\label{sec:state_of_the_art}

The following works can be highlighted regarding smart cities' mobile applications. Urbanopoly \cite{celino_urbanopoly_2012} presents an app for smartphones which combines Human Computation, \textit{gamification} and Linked Open Data to verify, correct and gather data about tourism venues. To achieve this, Urbanopoly offers different games to the users, like quizzes, photo taking contests, etc. Similar to Urbanopoly, Urbanmatch \cite{celino_urbanmatch_2012} can be found, a game in which the user takes photos about some tourism venues, in order to be published as Linked Open Data by the system. Another work that uses Human Computation for movie-related data curation is Linked Movie Quiz\footnote{\url{http://lamboratory.com/hacks/ldmq/}}. In \cite{braun_collaborative_2007}, the authors present \textit{csxPOI}, an application that allows its users to \textit{collaboratively create, share, and modify semantically annotated POIs}. These \textit{semantic POIs} are modelled through a set of ontologies developed to fulfill this specific task; and published following the Linked Open Data principles. csxPOI allows users to create custom ontology classes, modelling new POI categories, and to establish subclass, superclass or equality relationships among them. In addition to create new classes, users can link these categories to concepts extracted from DBpedia\footnote{\url{http://dbpedia.org}}. In order to detect duplicate POIs, csxPOI clusters the available POIs with the aim of finding similarities among them.

As can be seen, the authors that work with user-generated Linked Open Data have to deal with duplication, missclasification, mismatching and data enrichment issues; and, as previously described, the end-user has arisen as the most important agent in smart cities' environments. In the next sections we explain how the IES Cities project uses the Provenance Data Model to represent provenance information about user-generated data.

\section{IES Cities}
\label{sec:ies_cities}

`IES Cities'\footnote{\url{http://iescities.eu}}, is the last iteration in a chain of inter-related projects promoting user-centric and user-provided mobile services that exploit both Open Data and user-supplied data in order to develop innovative services.

The project encourages the re-use of already deployed sensor networks in European cities and the existing Open Government related datasets.It envisages smartphones as both a sensors-full device and a browser with increasing computational capabilities which is carried by almost every citizen.

IES Cities' main contribution is to design and implement an open technological platform to encourage the development of Linked Open Data based services, which will be later consumed by mobile applications. This platform will be deployed in 4 different European cities: Zaragoza and Majadahonda (Spain), Bristol (United Kingdom), and Rovereto (Italy), providing citizens the opportunity to get the most out of their city's data.

Remarkably, IES CITIES wants to analyse the impact that citizens may have on improving, extending and enriching the data these services will be based upon, as they will become leading actors of the new open data environment within the city. Nonetheless, the quality of the provided data may significantly vary from one citizen to another, not to mention the possibility of someone's interest in populating the system with fake data.

Thus, the need for evaluating the value and trust of the user contributed data requires the inclusion of a validation module \cite{hartig_publishing_2010}. In other words, we should be able to express special meta-information about the data submitted by IES Cities' users. The idea that a single way of representing and collecting provenance could be adopted internally by all systems does not seem to be realistic today, so the actual approaches are based on heterogeneous systems which export their provenance into a core data model, and applications that need to make sense of provenance information can then import it, process it, and reason over it \cite{ceolin_trust_2012}.

In addition, when considering user-provided data measures for data consolidation have to be considered. Contributions from one user have to be cross-validated with contributions from other users in order to avoid information duplication and foster validation of others' data. Thus, data contributions from different users presenting spatial, linguistic and semantic similarity should be clustered \cite{braun_collaborative_2010}. Before a user contributes with new data, other user's contributions at nearby locations should be shown to avoid recreating already existing data and encourage additions and enhancements to be applied to the existing data. After contributing with new data, the data providing user should be presented with earlier submitted similar contributions both in terms of contents and location in order to confirm whether their new contribution is actually a new contribution or it is amending an earlier existing one. In essence, aids before and after editing new entries have to be provided and a two phase commit process for user provided data should be put in place to ensure that contents of the highest quality are always added. Future work in IES CITIES will tackle these issues by providing REST interfaces to invoke services for clustering data entries and to retrieving related entries associated to a given one.

\section{Semantic representation of provenance}
\label{sec:use_cases}

To illustrate the semantic representation of trust and provenance data through Provenance Ontology and Uncertainty Provenance set of attributes, a use case is presented: 311 Bilbao. 311 Bilbao, uses Linked Open Data to get an overview of reports of faults in public infrastructure. From the data owner’s point of view, enrichment of their datasets by third parties, such as users of the 311 Bilbao application, revealed two problems: 1) the fact that data does not need to be approved before being published and that there is no mechanism to control the amount of data a citizen can add and 2) there is still the need for a way to differentiate the default trustworthiness of the different authors such as citizens and city councils. At the following code, the representation of the provenance of a user generated report is shown\footnote{The provenance data is represented using Provenance Notation (PROV-N). More information at \url{http://www.w3.org/TR/prov-n/}}:

\lstset{numbers=left, basicstyle=\ttfamily\scriptsize,}
\begin{lstlisting}
@prefix foaf: <http://xmlns.com/foaf/0.1/> .
@prefix prov: <http://www.w3.org/ns/prov#> .
@prefix iesc: <http://iescities.eu/ont#> .
@prefix up:   <http://users.ugent.be/~tdenies/up/> .
@prefix :     <http://bilbao.iescities.org#> .

entity(:report_23456, [ prov:value="The paper bin is
broken" ])
wasGeneratedBy(:report_23456, :reportActivity_23456)
wasAttributedTo(:report_23456, :jdoe)
wasInvalidatedBy(:report_23456, :invActivity_639,
2013-07-22T03:05:03)

activity(:reportActivity_23456, 2013-07-22T01:01:01,
2013-07-22T01:05:03)
wasAssociatedWith(:reportActivity_23456, :jdoe)

agent(:jdoe, [ prov:type='prov:Person', foaf:name=
"John Doe", foaf:mbox='<mailto:jdoe@example.org>' ])

entity(:report_23457, [ prov:value="It is incorrect,
another paper bin has replaced the old one, but 2
meters beyond" ])
wasAttributedTo(:report_23457, :jane)
wasDerivedFrom(:report_23457, :report_23456,
:invActivity_639, -, -, [ prov:type='prov:Revision' ])

activity(:invActivity_639, 2013-07-22T02:58:01,
2013-07-22T03:04:47)
wasAssociatedWith(:invActivity_639, :jane)

agent(:jane, [ prov:type='prov:Person', foaf:name=
"Jane", foaf:mbox='<mailto:jane@bilbao.iescities.org>'
 ])
actedOnBehalfOf(:jane, :bilbao_city_council)

agent(:bilbao_city_concil, [ prov:type=
'prov:Organization', foaf:name="Bilbao City Council"
])

\end{lstlisting}

On this piece of semantic information the \texttt{:report\_23456} resource represents the report made by the user. This report is identified by its own and unique URL and provides information about the user that has made it and which activity that has generated this report (lines 7-12). The \texttt{:reportActivity\_23456} shows details about the activity that generated the report, like when the user starts to reporting the issue and when has ended. At line 18 the information about "John Doe", the user that reported the fault, can be seen. In the example given, another user, Jane (lines 32-35), has revised the report made by John (lines 21-30). As the \texttt{actedOnBehalfOf} asserts, Jane is some kind of municipal worker of Bilbao City Council (line 37). As Jane's report has more authority agains John's report, John's report is invalidated as \texttt{wasInvalidatedBy} asserts. Allowing the semantic descriptions of the provenance of the reports made at 311 Bilbao app, the data generated by a concrete user can be reached through SPARQL \cite{prudhommeaux_sparql_2008} language queries.





Timeliness is another metric to be taken into account to evaluate data quality, and can be defined as the the up-to-date degree of a data item in relation with the task at hand. We propose and adaption of \cite{Hartig09usingweb} formula to measure timeliness, based on the work described in \cite{Ballou:1998:MIM:291329.291335}:

\begin{equation}
    timeliness = (max(1-\frac{currency}{volatility}), 0)^{sensitivity}
\end{equation}

where \emph{currency} is the difference between the time data is presented to the user and the time it was reported to the system. \emph{Volatility} refers to the maximum amount of time a given report time should be active (for example, if a broken street lamp is reported, it should be repaired within a month at most), and \emph{sensitivity} may change its value by observing the updates made over the status of the report: it would adopt a high value for data being constantly updated, and a low value for data that does not change often.


\section{Provenance data based reliability}
\label{sec:provenance}

There exist some approaches on how to calculate trust in semantic web using provenance information. IWTrust \cite{zaihrayeu2005iwtrust} uses provenance in the trust component of an answering engine, in which a trust value for answers is measured based on the trust in sources and in users. In \cite{golbeck2006combining} provenance data is used to evaluate the reliability of users based on trust relationships within a social network. \cite{Hartig09usingweb} presents an assesment method for evaluating the quality of data on the Web using provenance graphs, and provides a way to calculate trust values based on timeliness. In \cite{CeolinGHNF12} the authors propose generic procedures for computing reputation and trust assessments based on provenance information. 

In \cite{gil2007towards} the authors identify 19 parameters that affect how users determine trust in content provided by web information sources, such as the authority of the creator of the information or the popularity and recency of that information, among others. Based on these factors, we have built a generic model for the measurement of a trust value in the context of IES Cities, in which the trust according to each factor is calculated independently:
%
\begin{equation}
trust(report) = \frac{\sum_{p=[auth, agree...]}^{n} \alpha_p * trust_p(report)}{n}
\end{equation}
%
where \textit{p} is the measured property and \textit{n} is the total number of measured properties. {$\alpha$} is a value between 0 and 1 to denote the relevance of this property, making the measure based on a certain property more or less relevant. \textit{trust\textsubscript{p}} is a function that returns a value between 0 and 1 determining the trust of a given report according to a certain property.

Both the $\alpha$ values and the \textit{trust\textsubscript{p}} functions can be defined by the developers using IES Cities platform, because both of them are dependant on the context and the need of the application domain.

To clarify, we are using this model in the 311 Bilbao use case. To that end, we have selected the most relevant trust-properties concerning our use case:

\textbf{Authority:} Role stuff.
\begin{equation}
score_{authority} = \left\{\begin{matrix}
0 & if user \neq admin \\
1 & if user = admin
\end{matrix}\right.
\end{equation}

\textbf{Popularity:} Visits and stuff proposed by Diego.

\textbf{Agreement:} Reports saying the same thing.

\textbf{Recommendation:} Sacha's +1 / -1.

\textbf{Provenance:} Reputation of the user.

\textbf{Recency / Timeliness:} Timeliness can be defined as the the up-to-date degree of a data item in relation with the task at hand. We propose and adaption of \cite{Hartig09usingweb} formula to measure timeliness, based on the work described in \cite{Ballou:1998:MIM:291329.291335}:
%
\begin{equation}
    timeliness = (max(1-\frac{currency}{volatility}), 0)^{sensitivity}
\end{equation}
%
where \emph{currency} is the difference between the time data is presented to the user and the time it was reported to the system. \emph{Volatility} refers to the maximum amount of time a given report time should be active (for example, if a broken street lamp is reported, it should be repaired within a month at most), and \emph{sensitivity} may change its value by observing the updates made over the status of the report: it would adopt a high value for data being constantly updated, and a low value for data that does not change often.

\textbf{Context:} Geographical distance.
\begin{equation}
score_{distance} = \frac{1}{geodistance(loc_{report}, loc_{reportedplace})}
\end{equation}

\section{Conclusions and future work}
\label{sec:conclusions}


\bibliographystyle{abbrv}
\bibliography{immoa2013}

\end{document}
