\section{IES Cities}
\label{sec:ies_cities}

'IES Cities'\footnote{\url{http://iescities.eu}}, is the last iteration in a chain of inter-related projects promoting user-centric and user-provided mobile services that exploit both open data and user-supplied data in order to develop innovative services.

The project encourages the re-use of already deployed sensor networks in European cities, the existing Open Government related datasets, envisaging smartphones as both a sensor-full device in each citizen's pocket and a browser with increasing computation capacities to provide smart data to their owners.

IES Cities' main contribution is to design and implement an open technological platform to encourage the development of Linked Open Data based services, which will be later consumed by mobile applications. This platform will be deployed in 4 different European cities: Zaragoza and Majadahonda (Spain), Bristol (United Kingdom), and Rovereto (Italy), providing citizens the opportunity to get the most out of their city's data.

It is worth mentioning that no project before has considered so deeply the impact citizens may have on improving, extending and enriching the data these services will be based upon, as they will become leading actors of the new open data environment within the city. Nonetheless, the quality of the provided data may significantly vary from one citizen to another, not to mention the possibility of someone's interest in populating the system with fake data.

Thus, the need for evaluating the value and trust of the data requires the inclusion of a validation module\cite{hartig_publishing_2010}. In other words, we should be able to express special meta-information about the data delivered by IES Cities' users. The idea that a single way of representing and collecting provenance could be adopted internally by all systems does not seem to be realistic today, so the actual approaches are based on heterogeneous systems which export their provenance into a core data model, and applications that need to make sense of provenance information can then import it, process it, and reason over it\cite{ceolin2012trust}.
