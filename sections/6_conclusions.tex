\section{Conclusions and future work}
\label{sec:conclusions}

The proposed approach in this article will allow to evaluate the provenance of user-submitted data in IES Cities' platform. Previously explained metrics will measure data trustworthiness level, providing an extra confidence layer in the project's framework. Involved city councils and site administrators will be able to query data quality through SPARQL queries, retrieving only those results with a confidence level above a parameterised threshold.

The evaluation and validation of the proposed metrics against other implementations following the PROV-O ontology will be left for a future iteration on IES Cities, aggregating other significant metrics should they improve the provenance of the generated data.
