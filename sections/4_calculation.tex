\section{Provenance data based reliability}
\label{sec:provenance}

There exist some approaches on how to calculate trust in semantic web using provenance information. IWTrust \cite{zaihrayeu2005iwtrust} uses provenance in the trust component of an answering engine, in which a trust value for answers is measured based on the trust in sources and in users. In \cite{golbeck2006combining} provenance data is used to evaluate the reliability of users based on trust relationships within a social network. \cite{Hartig09usingweb} present an assesment method for evaluating the quality of data on the Web using provenance graphs, and provides a way to calculate trust values based on timeliness. In \cite{CeolinGHNF12} the authors propose generic procedures for computing reputation and trust assessments based on provenance information.

In \cite{gil2007towards} the authors identify some factors that affect how users determine trust in content provided by web information sources. We have selected some of these parameters according to r


\begin{equation}
confidence = confidence_{assertion} * confidence_{content}
\end{equation}

\begin{equation}
confidence_{content} = \frac{\sum_{p=1}^{n} \alpha_p * score_p(report)}{n}
\end{equation}

\begin{equation}
score_{authority} = \left\{\begin{matrix}
0 & if user \neq admin \\
1 & if user = admin
\end{matrix}\right.
\end{equation}

\begin{equation}
score_{distance} = \frac{1}{geodistance(loc_{report}, loc_{reportedplace})}
\end{equation}


Timeliness is another metric to be taken into account to evaluate data quality, and can be defined as the the up-to-date degree of a data item in relation with the task at hand. We propose and adaption of \cite{Hartig09usingweb} formula to measure timeliness, based on the work described in \cite{Ballou:1998:MIM:291329.291335}:

\begin{equation}
    timeliness = (max(1-\frac{currency}{volatility}), 0)^{sensitivity}
\end{equation}

where \emph{currency} is the difference between the time data is presented to the user and the time it was reported to the system. \emph{Volatility} refers to the maximum amount of time a given report time should be active (for example, if a broken street lamp is reported, it should be repaired within a month at most), and \emph{sensitivity} may change its value by observing the updates made over the status of the report: it would adopt a high value for data being constantly updated, and a low value for data that does not change often.
