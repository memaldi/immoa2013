\section{Related Work}
\label{sec:state_of_the_art}

The following works can be highlighted regarding smart cities' mobile applications. Urbanopoly \cite{celino_urbanopoly_2012} presents an app for smartphones which combines Human Computation, \textit{gamification} and Linked Open Data to verify, correct and gather data about tourism venues. To achieve this, Urbanopoly offers different games to the users, like quizzes, photo taking contests, etc. Similar to Urbanopoly, Urbanmatch \cite{celino_urbanmatch_2012} can be found, a game in which the user takes photos about some tourism venues, in order to be published as Linked Open Data by the system. Another work that uses Human Computation for movie-related data curation is Linked Movie Quiz\footnote{\url{http://lamboratory.com/hacks/ldmq/}}. In \cite{braun_collaborative_2007}, the authors present \textit{csxPOI}, an application that allows its users to \textit{collaboratively create, share, and modify semantically annotated POIs}. These \textit{semantic POIs} are modelled through a set of ontologies developed to fulfill this specific task; and published following the Linked Open Data principles. csxPOI allows users to create custom ontology classes, modelling new POI categories, and to establish subclass, superclass or equality relationships among them. In addition to create new classes, users can link these categories to concepts extracted from DBpedia \cite{auer_dbpedia:_2007}. In order to detect duplicate POIs, csxPOI clusters the available POIs with the aim of finding similarities among them.

As can be seen, the authors that work with user-generated Linked Open Data have to deal with duplication, missclasification, mismatching and data enrichment issues; and, as previously described, the end-user has arisen as the most important agent in smart cities' environments. In the next sections we explain how the IES Cities project uses the Provenance Data Model to represent provenance information about user-generated data.
