\section{Related Work}
\label{sec:state_of_the_art}

Regarding to smart cities' apps that use user generated content, we highlight the following works. Urbanopoly \cite{celino_urbanopoly_2012} presents an app for smartphones which combines Human Computation, \textit{gamification} and Linked Open Data to verify, correct and collect data about tourism venues. To achieve this, Urbanopoly offers different games to the users, like quizzes, photo taking, etc. Similar to Urbanopoly, we can find Urbanmatch \cite{celino_urbanmatch_2012}. Urbanmatch presents a game in which the user takes photos about some tourism venues, to be published as Linked Open Data by the system. Another work that uses Human Computation for movie-related data curation is Linked Movie Quiz\footnote{\url{http://lamboratory.com/hacks/ldmq/}}. In \cite{braun_collaborative_2007}, the authors present \textit{csxPOI}, an application to allow its users to \textit{collaboratively create, share, and modify semantic POIs}. These \textit{semantic POIs} are modelled through a set of ontologies, developed for fulfilling of this specific task; and published as Linked Open Data. csxPOI allows users to create custom ontology classes, modelling new POI categories, and to establish subclass, superclass or equality relationships among them. In addition to create new classes, users can link these categories to a concept extracted from DBPedia \cite{auer_dbpedia:_2007}. In order to detect duplicate POIs, csxPOI clusters these POIs with the aim of finding similarities among them.

As can be seen, the authors that work with user-generated Linked Open Data have to deal with some replication, mismatching and data enrichment issues; and, as we have described before, the user is the most important agent in smart cities. In the next sections we explain how IES Cities project uses Provenance Data Model to represent provenance information about user-generated data.
