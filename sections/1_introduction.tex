\section{Introduction}
\label{sec:introduction}

According to the ``Apps for Smart Cities Manifesto''\footnote{\url{http://www.appsforsmartcities.com/?q=manifesto}}, smart city applications could be sensible, connectable, accessible, ubiquitous, sociable, sharable and visible/augmented. It is not a coincidence that all of these features can be found in a standard smartphone: the popularity of these devices makes them the most suitable device to ensure access to the services provided by smart cities. As one of the main features of the smart cities is the participation of the citizens in their governance, it is not unusual that these citizens generate and share their own data through their smartphones. Reviewing the literature, we can find many examples of apps that deal with user generated data like Urbanopoly \cite{celino_urbanopoly_2012}, Urbanmatch \cite{celino_urbanmatch_2012} or the popular apps related to the 311 service in cities like Calgary, Minneapolis, Baltimore or San Diego, all of them available in Google Play\footnote{\url{https://play.google.com}}. The IES Cities project goes one step beyond, providing an entire architecture to allow citizens to develop apps based on Linked Open Data \cite{bizer_linked_2009} provided by government, through user-friendly JSON APIs. All of these works that manage user-generated data have the same worry about these data: are they reliable? How can we know if can a given user and, consequently, the data generated by him/her can be trusted? Recently, the W3C has created the PROV Data Model \cite{moreau_prov-dm:_2012}, for provenance interchange on the Web. This PROV Data Model describes the entities, activities and people involved in the creation of a piece of data, allowing the consumer to evaluate the reliability of the data based on the their provenance information. Furthermore, PROV was deliberately kept extensible, allowing various extended concepts and custom attributes to be used. For example, the Uncertainty Provenance (UP) \cite{de_nies_modeling_2013} set of attributes can be used to model the uncertainty of data, aggregated from heterogeneously divided trusted and untrusted sources, or with varying confidence. On this paper, we present how IES Cities platform integrates PROV Data Model and the related PROV-O ontology \cite{lebo_prov-o:_2013}, and Uncertainty Provenance set of attributes, allowing the exchange of provenance information about user-generated data in the context of smart cities. The final aim is to enrich the knowledge gathered about a city not only with government-provided or networked sensors’ provided data, but also with high quality and trustable data coming from the citizens themselves.

The remaining of the paper is organized as follows: in Section \ref{sec:state_of_the_art} the current state of the art on apps that deal with user data in the context of smart cities is presented. Section \ref{sec:ies_cities} outlines the main concepts about IES Cities project. Sections \ref{sec:provenance} and \ref{sec:use_cases} describes the metrics to calculate the reliability of the data and its semantic representation, through a use cases, respectively. Finally, in Section \ref{sec:conclusions} the conclusions and the future work are presented.
